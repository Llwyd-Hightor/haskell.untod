% untod.tex
% Copyright © 2020 Brent Longborough
% -----------------------------------------------------
\documentclass[a4paper,10pt,twoside,openany]{memoir}
% =====================================================
\usepackage[british]{babel}
\selectlanguage{british}
\usepackage[style=iso]{datetime2}
\usepackage[pcount,grumpy,mark,markifdirty]{gitinfo2}
\usepackage{tgpagella}
\usepackage{fontspec}
\setmainfont[Numbers={Proportional,OldStyle},Ligatures=TeX]{TeX Gyre Pagella}
\setsansfont[Numbers={Proportional,OldStyle},Ligatures=TeX]{TeX Gyre Adventor}
\setmonofont{Consolas}
\usepackage{enumitem}
\setlist[description]{%
    format=\ttfamily\bfseries,
    style=nextline,
    leftmargin=5em,
    itemsep=0.5\onelineskip}
% - - - - - - - - - - - - - - - - - - - - - - - - -
\definecolor{IBMAlert20}{HTML}{f1c21b}
\definecolor{IBMAlert40}{HTML}{ff832b}
\definecolor{IBMAlert50}{HTML}{24a148}
\definecolor{IBMAlert60}{HTML}{da1e28}
\definecolor{IBMBlack}{HTML}{000000}
\definecolor{IBMBlue100}{HTML}{001141}
\definecolor{IBMBlue10}{HTML}{edf5ff}
\definecolor{IBMBlue20}{HTML}{d0e2ff}
\definecolor{IBMBlue30}{HTML}{a6c8ff}
\definecolor{IBMBlue40}{HTML}{78a9ff}
\definecolor{IBMBlue50}{HTML}{4589ff}
\definecolor{IBMBlue60}{HTML}{0f62fe}
\definecolor{IBMBlue70}{HTML}{0043ce}
\definecolor{IBMBlue80}{HTML}{002d9c}
\definecolor{IBMBlue90}{HTML}{001d6c}
\definecolor{IBMCoolGrey100}{HTML}{121619}
\definecolor{IBMCoolGrey10}{HTML}{f2f4f8}
\definecolor{IBMCoolGrey20}{HTML}{dde1e6}
\definecolor{IBMCoolGrey30}{HTML}{c1c7cd}
\definecolor{IBMCoolGrey40}{HTML}{a2a9b0}
\definecolor{IBMCoolGrey50}{HTML}{878d96}
\definecolor{IBMCoolGrey60}{HTML}{697077}
\definecolor{IBMCoolGrey70}{HTML}{4d5358}
\definecolor{IBMCoolGrey80}{HTML}{343a3f}
\definecolor{IBMCoolGrey90}{HTML}{21272a}
\definecolor{IBMCyan100}{HTML}{061727}
\definecolor{IBMCyan10}{HTML}{e5f6ff}
\definecolor{IBMCyan20}{HTML}{bae6ff}
\definecolor{IBMCyan30}{HTML}{82cfff}
\definecolor{IBMCyan40}{HTML}{33b1ff}
\definecolor{IBMCyan50}{HTML}{1192e8}
\definecolor{IBMCyan60}{HTML}{0072c3}
\definecolor{IBMCyan70}{HTML}{00539a}
\definecolor{IBMCyan80}{HTML}{003a6d}
\definecolor{IBMCyan90}{HTML}{012749}
\definecolor{IBMGrey100}{HTML}{161616}
\definecolor{IBMGrey10}{HTML}{f4f4f4}
\definecolor{IBMGrey20}{HTML}{e0e0e0}
\definecolor{IBMGrey30}{HTML}{c6c6c6}
\definecolor{IBMGrey40}{HTML}{a8a8a8}
\definecolor{IBMGrey50}{HTML}{8d8d8d}
\definecolor{IBMGrey60}{HTML}{6f6f6f}
\definecolor{IBMGrey70}{HTML}{525252}
\definecolor{IBMGrey80}{HTML}{393939}
\definecolor{IBMGrey90}{HTML}{262626}
\definecolor{IBMGreen100}{HTML}{071908}
\definecolor{IBMGreen10}{HTML}{defbe6}
\definecolor{IBMGreen20}{HTML}{a7f0ba}
\definecolor{IBMGreen30}{HTML}{6fdc8c}
\definecolor{IBMGreen40}{HTML}{42be65}
\definecolor{IBMGreen50}{HTML}{24a148}
\definecolor{IBMGreen60}{HTML}{198038}
\definecolor{IBMGreen70}{HTML}{0e6027}
\definecolor{IBMGreen80}{HTML}{044317}
\definecolor{IBMGreen90}{HTML}{022d0d}
\definecolor{IBMMagenta100}{HTML}{2a0a18}
\definecolor{IBMMagenta10}{HTML}{fff0f7}
\definecolor{IBMMagenta20}{HTML}{ffd6e8}
\definecolor{IBMMagenta30}{HTML}{ffafd2}
\definecolor{IBMMagenta40}{HTML}{ff7eb6}
\definecolor{IBMMagenta50}{HTML}{ee5396}
\definecolor{IBMMagenta60}{HTML}{d12771}
\definecolor{IBMMagenta70}{HTML}{9f1853}
\definecolor{IBMMagenta80}{HTML}{740937}
\definecolor{IBMMagenta90}{HTML}{510224}
\definecolor{IBMPurple100}{HTML}{1c0f30}
\definecolor{IBMPurple10}{HTML}{f6f2ff}
\definecolor{IBMPurple20}{HTML}{e8daff}
\definecolor{IBMPurple30}{HTML}{d4bbff}
\definecolor{IBMPurple40}{HTML}{be95ff}
\definecolor{IBMPurple50}{HTML}{a56eff}
\definecolor{IBMPurple60}{HTML}{8a3ffc}
\definecolor{IBMPurple70}{HTML}{6929c4}
\definecolor{IBMPurple80}{HTML}{491d8b}
\definecolor{IBMPurple90}{HTML}{31135e}
\definecolor{IBMRed100}{HTML}{2d0709}
\definecolor{IBMRed10}{HTML}{fff1f1}
\definecolor{IBMRed20}{HTML}{ffd7d9}
\definecolor{IBMRed30}{HTML}{ffb3b8}
\definecolor{IBMRed40}{HTML}{ff8389}
\definecolor{IBMRed50}{HTML}{fa4d56}
\definecolor{IBMRed60}{HTML}{da1e28}
\definecolor{IBMRed70}{HTML}{a2191f}
\definecolor{IBMRed80}{HTML}{750e13}
\definecolor{IBMRed90}{HTML}{520408}
\definecolor{IBMTeal100}{HTML}{081a1c}
\definecolor{IBMTeal10}{HTML}{d9fbfb}
\definecolor{IBMTeal20}{HTML}{9ef0f0}
\definecolor{IBMTeal30}{HTML}{3ddbd9}
\definecolor{IBMTeal40}{HTML}{08bdba}
\definecolor{IBMTeal50}{HTML}{009d9a}
\definecolor{IBMTeal60}{HTML}{007d79}
\definecolor{IBMTeal70}{HTML}{005d5d}
\definecolor{IBMTeal80}{HTML}{004144}
\definecolor{IBMTeal90}{HTML}{022b30}
\definecolor{IBMWarmGrey100}{HTML}{171414}
\definecolor{IBMWarmGrey10}{HTML}{f7f3f2}
\definecolor{IBMWarmGrey20}{HTML}{e5e0df}
\definecolor{IBMWarmGrey30}{HTML}{cac5c4}
\definecolor{IBMWarmGrey40}{HTML}{ada8a8}
\definecolor{IBMWarmGrey50}{HTML}{8f8b8b}
\definecolor{IBMWarmGrey60}{HTML}{736f6f}
\definecolor{IBMWarmGrey70}{HTML}{565151}
\definecolor{IBMWarmGrey80}{HTML}{3c3838}
\definecolor{IBMWarmGrey90}{HTML}{272525}
\definecolor{IBMWhite}{HTML}{ffffff}
\definecolor{IBMYellow100}{HTML}{2d2b07}
\definecolor{IBMYellow10}{HTML}{fffff1}
\definecolor{IBMYellow20}{HTML}{fffdd7}
\definecolor{IBMYellow30}{HTML}{fffab3}
\definecolor{IBMYellow40}{HTML}{fff983}
\definecolor{IBMYellow50}{HTML}{faf14d}
\definecolor{IBMYellow60}{HTML}{dad01e}
\definecolor{IBMYellow70}{HTML}{a29c19}
\definecolor{IBMYellow80}{HTML}{75700e}
\definecolor{IBMYellow90}{HTML}{524e04}
\definecolor{IBMOrange100}{HTML}{2d1807}
\definecolor{IBMOrange10}{HTML}{ffe6cd}
\definecolor{IBMOrange20}{HTML}{ffe9d7}
\definecolor{IBMOrange30}{HTML}{ffd4b3}
\definecolor{IBMOrange40}{HTML}{ffbb83}
\definecolor{IBMOrange50}{HTML}{fa9b4d}
\definecolor{IBMOrange60}{HTML}{da721e}
\definecolor{IBMOrange70}{HTML}{a25819}
\definecolor{IBMOrange80}{HTML}{753d0e}
\definecolor{IBMOrange90}{HTML}{522704}

\usepackage{listings}
\lstdefinestyle{base}{%
  aboveskip=0.75\baselineskip,
  belowskip=0.75\baselineskip,
  backgroundcolor={\color{white}},
  basicstyle=\footnotesize\ttfamily,
  emphstyle={\itshape},
  frame=tb,
  framerule=\heavyrulewidth,
  numbers=none,
  stepnumber=1,
  numbersep=5pt,
  numberstyle=\color{black},
  escapechar=!,
  upquote=true,
}
\lstdefinestyle{zterm}{%
  backgroundcolor={\color{IBMBlue100}},
  basicstyle={\color{IBMGreen30}\scriptsize\ttfamily},
  emphstyle={\itshape},
  frame=L,
  framerule=1pt,
  rulecolor={\color{IBMRed60}},
  numbers=none,
  stepnumber=1,
  numbersep=5pt,
  numberstyle=\color{black},
  moredelim={[is][\bfseries]{|}{|}}
}
\lstdefinestyle{zentry}{%
  backgroundcolor={\color{black}},
  basicstyle={\color{white}\footnotesize\ttfamily},
  emphstyle={\itshape},
  frame=none,
  resetmargins=true,
  escapechar=!,
}
% - - - - - - - - - - - - - - - - - - - - - - - - -
\setulmarginsandblock{0.11111\paperwidth}{0.22222\paperwidth}{*}
\setlrmarginsandblock{0.11111\paperwidth}{0.22222\paperwidth}{*}
\setheadfoot{1.2\baselineskip}{0.0849\paperwidth}
\setmarginnotes{0.125\foremargin}{0.75\foremargin}{\onelineskip}
\setheaderspaces{*}{*}{0.618}
\checkandfixthelayout[fixed]
% \makepagenote
% \continuousnotenums
% \notepageref
% \foottopagenote
% \renewcommand*{\printpageinnotes}[1]{%
%   (p.\pageref{#1})\space}
% \renewcommand\printpageinnoteshyperref[1]{%
%   (p.\pageref*{#1})\space}
% \renewcommand*{\pagenotesubhead}[3]{%
%   \subsubsection*{#1: #3}}
\tightlists
\chapterstyle{bringhurst}
\pagestyle{empty}
\aliaspagestyle{chapter}{empty}
\settocdepth{subsection}
\setsecnumdepth{none}
\newcommand{\bpara}[1]{\par\vspace{\beforeparaskip}\noindent\textbf{#1}\,}
\newcommand{\rpara}[1]{\par\noindent\textbf{#1}\,}
\newcommand{\dark}[1]{\texttt\textbf{{#1}}}
\newcommand{\sfit}[1]{\textit{#1}}
\newcommand{\git}{\sfit{git}}
\newcommand*{\emailat}{@}
\newcommand{\opname}{\sfit{untod}}
\newcommand{\tpname}{\sfit{untod}}
\newcommand{\tpfname}{\textsf{untod.sty}}
\newcommand{\metaname}{\texttt{\ginname}}
\newcommand{\metapath}{\texttt{.git/\ginname}}
% -----------------------------------------------------
\usepackage[%
	bookmarksnumbered,
	bookmarksopen,
	linktocpage,
	]{hyperref}
\hypersetup{
   pdfauthor={Brent Longborough},
   pdftitle={The untod package: git metadata for LaTeX},
   pdfkeywords={git;metadata;dvcs},
}
\begin{document}
\frontmatter
% -----------------------------------------------------
\title{%
	~\\[2\baselineskip]
	\Huge \tpfname\\[2ex]%
	\Large A Swiss Army knife for System/Z and other clocks
	}
\author{Brent Longborough}
\date{\DTMenglishmonthname{\DTMfetchmonth{gitdate}} \DTMfetchyear{gitdate}}
\maketitle

{\centering
Release:\gitReln\ (\gitAbbrevHash)\\
}
% -----------------------------------------------------
\thispagestyle{empty}
\aliaspagestyle{chapter}{plain}
\clearforchapter
\tableofcontents*
% -----------------------------------------------------
\mainmatter
\pagestyle{giruled}
\aliaspagestyle{chapter}{giplain}
% -----------------------------------------------------
\chapter{Introduction}
\tpname\ provides a tool to convert among 
many different representations of a moment in time.

It accepts input as a System/Z TOD clock (hexadecimal),
date and time 
(conventional \textit{y-m-d} or julian \textit{y.d}),
an IPARS Perpetual Minute Clock (hexadecimal),
or a Unix seconds clock (decimal).

The output, written to standard output,
includes all of these values, 
along with a time convention, a time zone indication,
and a leap-seconds value.

Many options are provided:

\begin{description}
    \item[Timezones:] 
    Multiple time zone adjustments can be applied,
    either via command line options or 
    environment variables.
    \item[Clock conventions:] 
    The TOD clock may be adjusted according to 
    one of three conventions: UTC (with leap-seconds);
    TAI -- the International Atomic Clock 
    (without leap-seconds); or
    LORAN/IBM (without leap-seconds, 
    10 seconds ahead of LORAN).
    \item[Output formatting:] 
    As plain text or CSV;
    with headers; and with a run type annotation. 
    \item[Multiple sources of input:] 
    Values to be converted may be provided 
    on any or all of the command line, 
    the Windows clipboard, a named file, 
    or standard input.
    \item[Padding] TOD or Perpetual Minute Clock input 
    can be padded on the left or right.     
\end{description}
% - - - - - - - - - - - - - - - - - - - - - - - - - - -
\clearpage
\section{Sample output}
\begin{lstlisting}[style=zterm]
Ext       TOD              Date          Time        Zone     Julian   D    Perp        Unix      Leap
--- ----------------- : ---------- --------------- --------- -------- --- -------- -------------- ----
000 d84067a4 65ffc--- : 2020-07-21 16:33:46.341180 UTC+00:00 2020.203 Tue 01b5c6fd     1595352826 *+27 <== Date/Time
000 d84067a4 65ffc--- : 2020-07-21 17:33:46.341180 UTC+01:00 2020.203 Tue 01b5c739     1595356426 *+27 <== Date/Time
\end{lstlisting}
    
% -----------------------------------------------------
\chapter{Using the package}
% -----------------------------------------------------
\chapter{Etc}
% -----------------------------------------------------
\clearpage
\section{Acknowledgements}

The \href{http://tex.stackexchange.com}{\TeX.SE community}
has been a constant source of help, inspiration, and amazement.
In particular, I'd like to thank
\href{http://tex.stackexchange.com/users/73/joseph-wright}{Joseph Wright},
who rescued me from the jaws of the TeX parser by explaining
\textbackslash expandafter.

I'd also like to register my thanks to the owners of the packages on which
\tpname\ depends: datetime2, eso-pic, etoolbox, kvoptions, and xstring.

Many people have written to me kindly
to point out some of the defects in \opname, and to offer code.
I owe you all an apology for the amount of time that elapsed
from your suggestions to the making of \tpname.

In some cases, I have not taken up suggestions other than as food for thought,
in others used the code or suggestions directly, and,
in yet others, adapted.
I thank you all, especially for stimulating my thought processes,
and thus, hopefully,
helping to make \tpname\ a whole lot better than \opname.

My sincere thanks, too, to
Adrian Burd,
cedb12 (GitHub),
Maximilian Held,
Johannes Hoetzer,
Mikko Korpela,
Martin W Leidig,
Enrico Malizia,
Ken Mankoff,
Ryan Matlock,
Robbie Morrison,
Nik (gwdg nokta de),
Omid (gmail nokta com),
Sasaki~Suguru,
Tor\-bjørn~T (GitHub and TeX.SE),
and
Felix Wenger.

Special thanks to Karl Berry for helping me to
reduce my incompetence with \texttt{ctanify}.
And, of course, for \TeX{} Live and everything else.

Finally, but by no means least,
my thanks to the \textsc{Ctan} elves, and their dæmons,
particularly, in my case,
Ina Dau,
Manfred Lotz, 
Petra Rübe-Pugliese,
and 
Robin Fairbairns, 
for their infinite patience and unstinting
dedication to the \TeX\ community.

The failings, of course, I claim for myself.

% - - - - - - - - - - - - - - - - - - - - - - - - - - -
\clearpage
\section{Copyright \& licence}
Copyright \copyright\ \DTMfetchyear{gitdate}, Brent Longborough,
who has asserted his moral right
to be identified as the author of this work.

This work --- \tpname\ --- may be distributed and/or modified under the
conditions of the LaTeX Project Public License: either version 1.3
of this license, or (at your option) any later version.

The latest version of this license can be found
at the \LaTeX\ Project website,%
\footnote{(\url{http://www.latex-project.org/lppl.txt})}
and version 1.3 or later is part of all recent distributions of
\LaTeX.

This work has the LPPL maintenance status `maintained';
the Current Maintainer of this work is Brent Longborough.

This work consists of the files
untod.sty, gitexinfo.sty, untod.tex, untod.pdf,
untodtest.tex, post-xxx-sample.txt,

% - - - - - - - - - - - - - - - - - - - - - - - - - - -
\section{From the author}
Although my limitations as a \TeX nician
mean that I've implemented \tpname\ in a rather simplistic way
that needs some setup that is more complicated than I wanted,
I hope you find the package useful.
I'll be very happy to receive your comments by email.\\[\baselineskip]
Brent Longborough\\[\baselineskip]
\textsf{brent+ctancontrib (bei) longborough (punkt) org}\\
and at \href{http://tex.stackexchange.com/users/344/brent-longborough}{\TeX.SE}
% -----------------------------------------------------
\clearpage
\raggedright
\printpagenotes
\end{document}
