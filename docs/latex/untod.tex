% untod.tex
% Copyright © 2020 Brent Longborough
% -----------------------------------------------------
\documentclass[a4paper,12pt,oneside,openany]{memoir}
% =====================================================
\usepackage[british]{babel}
\selectlanguage{british}
\usepackage[style=iso]{datetime2}
\usepackage[pcount,grumpy,]{gitinfo2}
\usepackage{tgpagella}
\usepackage{fontspec}
\setmainfont[Numbers={Proportional,OldStyle},Ligatures=TeX]{TeX Gyre Pagella}
\setsansfont[Numbers={Proportional,OldStyle},Ligatures=TeX]{TeX Gyre Adventor}
\setmonofont{Consolas}
\newfontfamily\supersmall[Scale=0.61]{Consolas}
\usepackage{enumitem}
\setlist[description]{%
    format=\bfseries,
    style=nextline,
    leftmargin=3em,
    itemsep=0.5\onelineskip}
% - - - - - - - - - - - - - - - - - - - - - - - - -
\definecolor{IBMAlert20}{HTML}{f1c21b}
\definecolor{IBMAlert40}{HTML}{ff832b}
\definecolor{IBMAlert50}{HTML}{24a148}
\definecolor{IBMAlert60}{HTML}{da1e28}
\definecolor{IBMBlack}{HTML}{000000}
\definecolor{IBMBlue100}{HTML}{001141}
\definecolor{IBMBlue10}{HTML}{edf5ff}
\definecolor{IBMBlue20}{HTML}{d0e2ff}
\definecolor{IBMBlue30}{HTML}{a6c8ff}
\definecolor{IBMBlue40}{HTML}{78a9ff}
\definecolor{IBMBlue50}{HTML}{4589ff}
\definecolor{IBMBlue60}{HTML}{0f62fe}
\definecolor{IBMBlue70}{HTML}{0043ce}
\definecolor{IBMBlue80}{HTML}{002d9c}
\definecolor{IBMBlue90}{HTML}{001d6c}
\definecolor{IBMCoolGrey100}{HTML}{121619}
\definecolor{IBMCoolGrey10}{HTML}{f2f4f8}
\definecolor{IBMCoolGrey20}{HTML}{dde1e6}
\definecolor{IBMCoolGrey30}{HTML}{c1c7cd}
\definecolor{IBMCoolGrey40}{HTML}{a2a9b0}
\definecolor{IBMCoolGrey50}{HTML}{878d96}
\definecolor{IBMCoolGrey60}{HTML}{697077}
\definecolor{IBMCoolGrey70}{HTML}{4d5358}
\definecolor{IBMCoolGrey80}{HTML}{343a3f}
\definecolor{IBMCoolGrey90}{HTML}{21272a}
\definecolor{IBMCyan100}{HTML}{061727}
\definecolor{IBMCyan10}{HTML}{e5f6ff}
\definecolor{IBMCyan20}{HTML}{bae6ff}
\definecolor{IBMCyan30}{HTML}{82cfff}
\definecolor{IBMCyan40}{HTML}{33b1ff}
\definecolor{IBMCyan50}{HTML}{1192e8}
\definecolor{IBMCyan60}{HTML}{0072c3}
\definecolor{IBMCyan70}{HTML}{00539a}
\definecolor{IBMCyan80}{HTML}{003a6d}
\definecolor{IBMCyan90}{HTML}{012749}
\definecolor{IBMGrey100}{HTML}{161616}
\definecolor{IBMGrey10}{HTML}{f4f4f4}
\definecolor{IBMGrey20}{HTML}{e0e0e0}
\definecolor{IBMGrey30}{HTML}{c6c6c6}
\definecolor{IBMGrey40}{HTML}{a8a8a8}
\definecolor{IBMGrey50}{HTML}{8d8d8d}
\definecolor{IBMGrey60}{HTML}{6f6f6f}
\definecolor{IBMGrey70}{HTML}{525252}
\definecolor{IBMGrey80}{HTML}{393939}
\definecolor{IBMGrey90}{HTML}{262626}
\definecolor{IBMGreen100}{HTML}{071908}
\definecolor{IBMGreen10}{HTML}{defbe6}
\definecolor{IBMGreen20}{HTML}{a7f0ba}
\definecolor{IBMGreen30}{HTML}{6fdc8c}
\definecolor{IBMGreen40}{HTML}{42be65}
\definecolor{IBMGreen50}{HTML}{24a148}
\definecolor{IBMGreen60}{HTML}{198038}
\definecolor{IBMGreen70}{HTML}{0e6027}
\definecolor{IBMGreen80}{HTML}{044317}
\definecolor{IBMGreen90}{HTML}{022d0d}
\definecolor{IBMMagenta100}{HTML}{2a0a18}
\definecolor{IBMMagenta10}{HTML}{fff0f7}
\definecolor{IBMMagenta20}{HTML}{ffd6e8}
\definecolor{IBMMagenta30}{HTML}{ffafd2}
\definecolor{IBMMagenta40}{HTML}{ff7eb6}
\definecolor{IBMMagenta50}{HTML}{ee5396}
\definecolor{IBMMagenta60}{HTML}{d12771}
\definecolor{IBMMagenta70}{HTML}{9f1853}
\definecolor{IBMMagenta80}{HTML}{740937}
\definecolor{IBMMagenta90}{HTML}{510224}
\definecolor{IBMPurple100}{HTML}{1c0f30}
\definecolor{IBMPurple10}{HTML}{f6f2ff}
\definecolor{IBMPurple20}{HTML}{e8daff}
\definecolor{IBMPurple30}{HTML}{d4bbff}
\definecolor{IBMPurple40}{HTML}{be95ff}
\definecolor{IBMPurple50}{HTML}{a56eff}
\definecolor{IBMPurple60}{HTML}{8a3ffc}
\definecolor{IBMPurple70}{HTML}{6929c4}
\definecolor{IBMPurple80}{HTML}{491d8b}
\definecolor{IBMPurple90}{HTML}{31135e}
\definecolor{IBMRed100}{HTML}{2d0709}
\definecolor{IBMRed10}{HTML}{fff1f1}
\definecolor{IBMRed20}{HTML}{ffd7d9}
\definecolor{IBMRed30}{HTML}{ffb3b8}
\definecolor{IBMRed40}{HTML}{ff8389}
\definecolor{IBMRed50}{HTML}{fa4d56}
\definecolor{IBMRed60}{HTML}{da1e28}
\definecolor{IBMRed70}{HTML}{a2191f}
\definecolor{IBMRed80}{HTML}{750e13}
\definecolor{IBMRed90}{HTML}{520408}
\definecolor{IBMTeal100}{HTML}{081a1c}
\definecolor{IBMTeal10}{HTML}{d9fbfb}
\definecolor{IBMTeal20}{HTML}{9ef0f0}
\definecolor{IBMTeal30}{HTML}{3ddbd9}
\definecolor{IBMTeal40}{HTML}{08bdba}
\definecolor{IBMTeal50}{HTML}{009d9a}
\definecolor{IBMTeal60}{HTML}{007d79}
\definecolor{IBMTeal70}{HTML}{005d5d}
\definecolor{IBMTeal80}{HTML}{004144}
\definecolor{IBMTeal90}{HTML}{022b30}
\definecolor{IBMWarmGrey100}{HTML}{171414}
\definecolor{IBMWarmGrey10}{HTML}{f7f3f2}
\definecolor{IBMWarmGrey20}{HTML}{e5e0df}
\definecolor{IBMWarmGrey30}{HTML}{cac5c4}
\definecolor{IBMWarmGrey40}{HTML}{ada8a8}
\definecolor{IBMWarmGrey50}{HTML}{8f8b8b}
\definecolor{IBMWarmGrey60}{HTML}{736f6f}
\definecolor{IBMWarmGrey70}{HTML}{565151}
\definecolor{IBMWarmGrey80}{HTML}{3c3838}
\definecolor{IBMWarmGrey90}{HTML}{272525}
\definecolor{IBMWhite}{HTML}{ffffff}
\definecolor{IBMYellow100}{HTML}{2d2b07}
\definecolor{IBMYellow10}{HTML}{fffff1}
\definecolor{IBMYellow20}{HTML}{fffdd7}
\definecolor{IBMYellow30}{HTML}{fffab3}
\definecolor{IBMYellow40}{HTML}{fff983}
\definecolor{IBMYellow50}{HTML}{faf14d}
\definecolor{IBMYellow60}{HTML}{dad01e}
\definecolor{IBMYellow70}{HTML}{a29c19}
\definecolor{IBMYellow80}{HTML}{75700e}
\definecolor{IBMYellow90}{HTML}{524e04}
\definecolor{IBMOrange100}{HTML}{2d1807}
\definecolor{IBMOrange10}{HTML}{ffe6cd}
\definecolor{IBMOrange20}{HTML}{ffe9d7}
\definecolor{IBMOrange30}{HTML}{ffd4b3}
\definecolor{IBMOrange40}{HTML}{ffbb83}
\definecolor{IBMOrange50}{HTML}{fa9b4d}
\definecolor{IBMOrange60}{HTML}{da721e}
\definecolor{IBMOrange70}{HTML}{a25819}
\definecolor{IBMOrange80}{HTML}{753d0e}
\definecolor{IBMOrange90}{HTML}{522704}

\colorlet{bg}{IBMBlack}
\colorlet{fg}{IBMCoolGrey20}
\colorlet{utc}{IBMGreen40}
\colorlet{lor}{IBMBlue50}
\colorlet{tai}{IBMTeal40}
\colorlet{head}{IBMAlert20}
\colorlet{note}{IBMRed60}
\colorlet{rule}{IBMBlue50}
\colorlet{num}{IBMBlack}
\usepackage{listings}
\lstdefinestyle{base}{%
backgroundcolor={\color{bg}},
basicstyle={\color{fg}\ttfamily},
emphstyle={\itshape},
frame=none,
framexleftmargin=5pt,
framexrightmargin=5pt,
numbers=left,
stepnumber=1,
numbersep=8pt,
numberstyle={\color{num}\ttfamily\small},
escapechar=!,
upquote=true,
morecomment=[f][\color{rule}]{---},
morecomment=[f][\color{rule}]{===},
morecomment=[l][\color{note}]{<==},
morekeywords={[1]Ext,TOD,Date,Time,Zone,
Julian,D,Perp,Unix,Leap},
keywordstyle={[1]\color{head}},
morekeywords={[2]UTC,},
keywordstyle={[2]\color{utc}},
morekeywords={[3]LOR,},
keywordstyle={[3]\color{lor}},
morekeywords={[4]TAI,},
keywordstyle={[4]\color{tai}},
}
\lstdefinestyle{plain}{%
  style=base,
  basicstyle={\color{fg}\ttfamily},
  keywords=[1]{},
  keywords=[2]{},
  keywords=[3]{},
  keywords=[4]{},
  keywords=[5]{},
  numbers=none,
}
\lstdefinestyle{xterm}{%
  style=base,
}
\lstdefinestyle{xtermsmall}{%
  style=base,
  resetmargins=true,
  basicstyle={\color{IBMWhite}\supersmall},
}
% - - - - - - - - - - - - - - - - - - - - - - - - -
\setulmarginsandblock{0.11111\paperwidth}{0.22222\paperwidth}{*}
\setlrmarginsandblock{1.5cm}{1.5cm}{*}
\setheadfoot{1.2\baselineskip}{0.0849\paperwidth}
\setheaderspaces{*}{*}{0.618}
\checkandfixthelayout[fixed]
\tightlists
\chapterstyle{bringhurst}
\pagestyle{empty}
\aliaspagestyle{chapter}{empty}
\settocdepth{subsection}
\setsecnumdepth{none}
\newcommand{\ddash}{-\rule{0.1pt}{0pt}-}
\newcommand{\bpara}[1]{\par\vspace{\beforeparaskip}\noindent\textbf{#1}\,}
\newcommand{\rpara}[1]{\par\noindent\textbf{#1}\,}
\newcommand{\dark}[1]{\texttt\textbf{{#1}}}
\newcommand{\sfit}[1]{\textit{#1}}
\newcommand{\git}{\sfit{git}}
\newcommand*{\emailat}{@}
\newcommand{\opname}{\sfit{untod}}
\newcommand{\tpname}{\sfit{untod}}
\newcommand{\tpfname}{\textsf{untod.sty}}
\newcommand{\metaname}{\texttt{\ginname}}
\newcommand{\metapath}{\texttt{.git/\ginname}}
% -----------------------------------------------------
\usepackage[%
	bookmarksnumbered,
	bookmarksopen,
	linktocpage,
	]{hyperref}
\hypersetup{
   pdfauthor={Brent Longborough},
   pdftitle={The untod package: git metadata for LaTeX},
   pdfkeywords={git;metadata;dvcs},
}

\begin{document}
\frontmatter
% -----------------------------------------------------
\title{%
	~\\[2\baselineskip]
	\Huge \tpfname\\[2ex]%
	\Large A Swiss Army knife for System/Z and other clocks
	}
\author{Brent Longborough}
\date{\DTMenglishmonthname{\DTMfetchmonth{gitdate}} \DTMfetchyear{gitdate}}
\maketitle

{\centering
Release:\gitReln\ (\gitAbbrevHash)\\
}
% -----------------------------------------------------
\thispagestyle{empty}
\aliaspagestyle{chapter}{plain}
\clearforchapter
\tableofcontents*
% -----------------------------------------------------
\mainmatter
\pagestyle{giruled}
\aliaspagestyle{chapter}{giplain}
% -----------------------------------------------------
\chapter{Introduction}
\tpname\ provides a tool to convert among
many different representations of a moment in time.

It accepts input as a System/Z TOD clock (hexadecimal),
date and time
(conventional \textit{y-m-d} or julian \textit{y.d}),
an IPARS Perpetual Minute Clock (hexadecimal),
or a Unix seconds clock (decimal).

The output, written to standard output,
includes all of these values,
along with a time convention, a time zone indication,
and a leap-seconds value.

In addition to a choice of input type,
many options are provided:

\begin{description}
    \item[Timezones:]
    Multiple time zone adjustments can be applied,
    either via command line options or
    environment variables.

    \item[Clock conventions:]
    One of three conventions may be used for
    the TOD clock: UTC (with leap-seconds);
    TAI -- the International Atomic Clock
    (without leap-seconds); or
    LORAN/IBM (without leap-seconds,
    10 seconds ahead of LORAN).

    \item[Output formatting:]
    As plain text or CSV;
    with headers; and with a run type annotation.

    \item[Multiple sources of input:]
    Values to be converted may be provided
    on any or all of the command line,
    the Windows clipboard, a named file,
    or standard input.

    \item[Padding:] TOD or Perpetual Minute Clock input
    can be padded on the left or right.

    \item[Metadata:] Version and help displays

\end{description}
% - - - - - - - - - - - - - - - - - - - - - - - - - - -
\clearpage
\section{Sample output}
\begin{lstlisting}[style=xtermsmall,
  ]
  ======================================================================================================
  Version: 0.1.1.0 (#153f3a0* -- Fri Jul 24 19:47:11 2020 +0100)
  (develop) Fix splitting, tidy cruft
  ------------------------------------------------------------------------------------------------------
  Ext       TOD              Date          Time        Zone     Julian   D    Perp        Unix      Leap
  --- ----------------- : ---------- --------------- --------- -------- --- -------- -------------- ----
  000 d84587ff aa08f--- : 2020-07-25 18:25:09.358542 UTC+00:00 2020.207 Sat 01b5dded     1595705109 *+27
  000 d84587ff aa08f--- : 2020-07-25 19:25:09.358542 UTC+01:00 2020.207 Sat 01b5de29     1595708709 *+27
  000 d84587ff aa08f--- : 2020-07-25 20:25:09.358542 UTC+02:00 2020.207 Sat 01b5de65     1595712309 *+27
  000 d84587ff aa08f--- : 2020-07-25 20:55:09.358542 UTC+02:30 2020.207 Sat 01b5de83     1595714109 *+27
  ======================================================================================================
untod -z --lzone=1 --azone=-8 -d 2020-07-25@19:19:45.545965
000 d845ff7d e67ad--- : 2020-07-26 03:19:45.545965 UTC+00:00 2020.208 Sun 01b5dde7     1595704785 *+27 <== Date/Time
000 d845ff7d e67ad--- : 2020-07-25 19:19:45.545965 UTC-08:00 2020.207 Sat 01b5dc07     1595675985 *+27 <== Date/Time
000 d845ff7d e67ad--- : 2020-07-26 04:19:45.545965 UTC+01:00 2020.208 Sun 01b5de23     1595708385 *+27 <== Date/Time
untod -l -z --lzone=1 --azone=-8 -d 2020-07-25@19:19:45.545965
000 d845ff64 26aed--- : 2020-07-26 03:19:45.545965 LOR+00:00 2020.208 Sun 01b5dde7     1595704785      <== Date/Time
000 d845ff64 26aed--- : 2020-07-25 19:19:45.545965 LOR-08:00 2020.207 Sat 01b5dc07     1595675985      <== Date/Time
000 d845ff64 26aed--- : 2020-07-26 04:19:45.545965 LOR+01:00 2020.208 Sun 01b5de23     1595708385      <== Date/Time
untod -t -z --lzone=1 --azone=-8 -d 2020-07-25@19:19:45.545965
000 d845ff5a 9d46d--- : 2020-07-26 03:19:45.545965 TAI+00:00 2020.208 Sun 01b5dde7     1595704785      <== Date/Time
000 d845ff5a 9d46d--- : 2020-07-25 19:19:45.545965 TAI-08:00 2020.207 Sat 01b5dc07     1595675985      <== Date/Time
000 d845ff5a 9d46d--- : 2020-07-26 04:19:45.545965 TAI+01:00 2020.208 Sun 01b5de23     1595708385      <== Date/Time
======================================================================================================
\end{lstlisting}
\clearpage
\begin{lstlisting}[style=plain,
  basicstyle={\color{fg}\ttfamily\footnotesize},]
  untod 0.1.1.0 - a Swiss Army knife for TOD and other clocks

  Usage: untod [(-d|--date) | (-o|--tod) | (-m|--pmc) | (-u|--unix) |
                 (-s|--seconds)] [-c|--clipboard] [--csv] [--headers] [-a|--annot]
               [-z|--zulu] [(-g|--gmt) | (-l|--loran) | (-t|--tai)]
               [--lpad | --rpad] [-i|--input <filename>] [--azone <offset>]
               [--lzone <offset>] [-v|--version] [<value...>] [-x]
    Converts among TOD, Date/Time, PARS Perpetual Minute Tick, Unix seconds, and
    20th century seconds for UTC, TAI or LORAN/IBM

  Available options:
    -d,--date                Convert from Date/Time (default)
    -o,--tod                 Convert from TOD
    -m,--pmc                 Convert from PMC
    -u,--unix                Convert from Unix seconds
    -s,--seconds             Convert from 20th Century seconds
    -c,--clipboard           Input values from clipboard
    --csv                    Output in CSV format
    --headers                Output column headers
    -a,--annot               Annotate plain output with run mode
    -z,--zulu                Also show Zulu offset
    -g,--gmt                 UTC with leap-seconds (default)
    -l,--loran               Ignore leap-seconds -- LORAN/IBM
    -t,--tai                 Ignore leap-seconds -- TAI (International Atomic
                             Clock
    --lpad                   Pad TOD with zeroes on left
    --rpad                   Pad TOD with zeroes on right (default is intelligent
                             padding)
    -i,--input <filename>    Input values from a file (- for STDIN)
    --azone <offset>         Alternative time offset ([-+]n.n) [env: UNTOD_AZONE=]
    --lzone <offset>         Override local time offset ([-+]n.n) [env:
                             UNTOD_LZONE=]
    -v,--version             Show version; more -v flags, more info
    <value...>               Values for conversion
    -h,--help                Show this help text
\end{lstlisting}

% -----------------------------------------------------
\chapter{Command line reference}
% - - - - - - - - - - - - - - - - - - - - - - - - - - -
\tpname\ is a conventional, Unix-like, command utility,
controlled by a series of options and input values
provided on the command line.
Any command line parameters not recognised as options
are values to be converted.
Multiple command line values can be converted.

% - - - - - - - - - - - - - - - - - - - - - - - - - - -
\section{Input type}

\tpname\ accepts one of five mutually exclusive options
to define the input data type for conversion:
\begin{description}[format=\ttfamily]
  \item[-d,--date] Convert from Date/Time

  This is the default input data type.
  Input values are treated as date and time stamps
  relative to the local time zone,
  in one of two input formats:
\begin{lstlisting}[style=plain]
yyyy-mm-dd@hh:mm:ss.ffffff
yyyy.jjj@hh:mm:ss.ffffff
\end{lstlisting}
  or `sensible' abbreviations of them.
  Where input values are abbreviated,
  they are padded on the right
  from one of two standard date patterns
  (with an example):
\begin{lstlisting}[style=plain]
1900-01-01@00:00:00.000000000
1900.001@00:00:00.000000000

    !2020-07!
+   !\color{IBMAlert20}1900-01!-01@00:00:00.000000000
=>  !2020-07!-01@00:00:00.000000000
\end{lstlisting}

  If no input values are provided,
  then the current date and time in the local time zone
  is used.

  Examples:
  \begin{lstlisting}[style=xtermsmall,numbers=none]
untod
000 d846712c 1e6cb--- : 2020-07-26 12:48:21.433867 UTC+01:00 2020.208 Sun 01b5e23c     1595771301 *+27
untod -d 2020-12-31
000 d90c6c1d 318c0--- : 2020-12-31 00:00:00.000000 UTC+01:00 2020.366 Thu 01b957fc     1609376400 *+27
  \end{lstlisting}

  \item[-o,--tod] Convert from TOD

  Input values are treated as 1-16
  hexadecimal character TOD clock values.
  Unless otherwise specified by
  the \texttt{--lpad} or \texttt{--rpad} options,
  described later,
  input values are padded to sixteen characters
  as follows:

  \begin{itemize}
    \item Values longer than 13 characters are padded
    with leading zeroes:\\[0.2\baselineskip]
    \texttt{ffffffffffffff} $ \Longrightarrow$
    \texttt{00f ffffffff fffff}

    \item Values of 13 or less characters starting with
    `c', `d', `e', or `f' are padded
    with three leading zeroes,
    and then to 16 places
    with trailing zeroes:\\[0.2\baselineskip]
    \texttt{d8} $ \Longrightarrow$
    \texttt{000 d8000000 00000}

    \item Other values of 13 or less characters are padded
    with two leading zeroes, and then to 16 places
    with trailing zeroes:\\[0.2\baselineskip]
    \texttt{11} $ \Longrightarrow$
    \texttt{001 10000000 00000}
  \end{itemize}

  Note that only the least significant 13 hexadecimal 
  characters (52 bits) are reflected by
  the hardware implementation of the TOD clock.
  The multiple-epoch facility extends this upwards
  by two (8 bits), extending the range
  from 2042 to 38,434.

  \item[-m,--pmc] Convert from PMC

  The IPARS Perpetual Minute Clock is a 32-bit 
  count of minutes since local midnight 
  at the start of January 3rd, 1966.
  Input values are treated as 1-8
  hexadecimal character minute clock values.
  Unless otherwise specified by
  the \texttt{--lpad} or \texttt{--rpad} options,
  described later,
  input values are padded to eight characters
  as follows:

  \begin{itemize}
    \item Values starting with `0'
    are padded on the right:\\[0.2\baselineskip]
    \texttt{01b} $ \Longrightarrow$
    \texttt{01b00000}

    \item All other values
    are padded on the left:\\[0.2\baselineskip]
    \texttt{d8} $ \Longrightarrow$
    \texttt{000000d8}

  \end{itemize}

  Note that the Perpetual Minute Clock is a local clock.
  Equal values will give identical dates and times, 
  independently of time zone, 
  but the resulting TOD and Unix clocks will vary 
  to reflect differences among timezones.

  The full 32-bit range of the Perpetual Minute Clock
  allows for dates and times up to the year 10,132.
  This is unlikely to be a practical limitation.

  \item[-u,--unix] Convert from Unix seconds


  \item[-s,--seconds] Convert from 20th Century seconds
\end{description}
% - - - - - - - - - - - - - - - - - - - - - - - - - - -
\section{Timezones}

\begin{description}[format=\ttfamily]
  \item[-z,--zulu]
    Also show Zulu offset
  \item[--azone \textit{offset}]
    Alternative time offset
    ([-+]n.n) [env: UNTOD\_AZONE=]
  \item[--lzone \textit{offset}]
    Local time offset
    ([-+]n.n) [env: UNTOD\_LZONE=]
\end{description}

% - - - - - - - - - - - - - - - - - - - - - - - - - - -
\section{Clock conventions}

\begin{description}[format=\ttfamily]
  \item[-g,--gmt  ]
    UTC with leap-seconds (default)
  \item[-l,--loran]
    Ignore leap-seconds -- LORAN/IBM
  \item[-t,--tai  ]
    Ignore leap-seconds --
    TAI -- International Atomic Clock
\end{description}

% - - - - - - - - - - - - - - - - - - - - - - - - - - -
\section{Output formatting}

\begin{description}[format=\ttfamily]
  \item[--csv]
  Output in CSV format
  \item[--headers]
  Output column headers
  \item[-a,--annot]
  Annotate plain output with run mode
\end{description}

% - - - - - - - - - - - - - - - - - - - - - - - - - - -
\section{Multiple sources of input}



\begin{description}[format=\ttfamily]
  \item[\textit{value...}]
    Values for conversion from command line
  \item[-c,--clipboard]
    Input values from clipboard
  \item[-i,--input \textit{path}]
    Input values from a file
    (- for STDIN)

\end{description}

% - - - - - - - - - - - - - - - - - - - - - - - - - - -
\section{Padding}

\begin{description}[format=\ttfamily]
  \item[--lpad]
  Pad TOD with zeroes on left
  \item[--rpad]
  Pad TOD with zeroes on right
  (default is intelligent padding)
\end{description}

% - - - - - - - - - - - - - - - - - - - - - - - - - - -
\section{Metadata}

\begin{description}[format=\ttfamily]
  \item[-v,--version]
    Show version;
    more -v flags, more info
  \item[-h,--help]
    Show help text
\end{description}

% -----------------------------------------------------
\chapter{Etc}
% -----------------------------------------------------
\clearpage

% - - - - - - - - - - - - - - - - - - - - - - - - - - -
\clearpage
\section{Copyright \& licence}
% - - - - - - - - - - - - - - - - - - - - - - - - - - -
\section{From the author}
% -----------------------------------------------------
\clearpage
\raggedright
\printpagenotes
\end{document}
